\iffalse
TODO:

- Magneten kaufen: http://www.amazon.de/Singende-Magnetsteine-Rattle-Snake-2er/dp/B00511MMY0/ref=sr_1_2?ie=UTF8&qid=1426673724&sr=8-2&keywords=zwitscher+magnete

- Datum

- Chirp center

- Simulation alt tab

Ablauf:
1. Experiment + Erklärung + Simulation
2. Analyse erklären
	a. Chirp, Peak, PeakPeak
	b. Trennung von Frequenzen
		1. Chirp Frequenz (wann die Peaks kommen)
		2. Peak Frequenz (die Frequenz eines einzelnen Peaks)
3. Analyse
	a. Chirp Analyse
		1. Periodendauer ((Zeitabstände) (1/f)) --> e Funktion, Faktor wegen damped oszilation
		2. Amplitude über Zeit siehe oben, gleicher Faktor (Ungenauigkeiten)
		3. Gleichsetzung (Seite 10)
			a. Ansätze von den einzelnen Faktoren
			b. Ergebnisse
		4. Verallgemeinerung
			a. why
			b. Energie Ansatz
			c. Graph
	b. Peak Analyse
		1. Frequenz eines Peaks (siehe Nacharbeitung)
\fi


\documentclass[12pt]{beamer}
\usetheme{Warsaw}
\usepackage[utf8]{inputenc}
\usepackage[german]{babel}
\usepackage{amsmath}
\usepackage{amsfonts}
\usepackage{amssymb}
\usepackage{graphicx}
\usepackage{media9}
\usepackage{moresize}
\usepackage{hyperref}
\usepackage{xfrac}
\author{Leonard Hackel und Niklas Schelten}
\title{5. PK - Ball Sound}
\subtitle{Wie entwickelt sich der Ton, der beim Zusammenstoß zweier Metallkugeln entsteht?}
\setbeamertemplate{navigation symbols}{}
\logo{\includegraphics[scale=0.028]{Bilder/Logo.png}} 
\institute{Herder Oberschule Berlin} 
\date{\today}
\subject{Physik} 

\newcommand{\multiplelines}[2][c]{\begin{tabular}[#1]{@{}l@{}}#2\end{tabular}}

%Frame numbers
\setbeamertemplate{footline}
{%
\begin{beamercolorbox}[wd=0.5\textwidth,ht=3ex,dp=1.5ex,leftskip=.5em,rightskip=.5em]{author in head/foot}%
\usebeamerfont{author in head/foot}%
\insertframenumber\ von \inserttotalframenumber\hfill\insertshortauthor%
\end{beamercolorbox}%
\vspace*{-4.5ex}\hspace*{0.5\textwidth}%
\begin{beamercolorbox}[wd=0.5\textwidth,ht=3ex,dp=1.5ex,left,leftskip=.5em]{title in head/foot}%
\usebeamerfont{title in head/foot}%
\insertshorttitle%
\end{beamercolorbox}%
}
%-------------------------------------------------------------------------------------------------
\begin{document}

\begin{frame}
\titlepage
\end{frame}

\begin{frame}
\tableofcontents
\end{frame}

%-------------------------------------------------------------------------------------------------
\section{Das Experiment}
\subsection{Vorführung}
\begin{frame}{Experiment}

\end{frame}

\subsection{Zusammensetzung des Tons}
\begin{frame}{\only<1-2>{Chirp}\only<3-4>{Peak}\only<5>{PeakPeak}}
\center{
\only<1>{\includemedia[addresource=Daten/Chirp.mp3, flashvars={source=Daten/Chirp.mp3&autoPlay=true}]
	{\includegraphics[scale=0.2]{Bilder/Chirp1.png}}{APlayer.swf}}
\only<2>{\includegraphics[scale=0.2]{Bilder/Chirp2.png}}
\only<3>{\includegraphics[scale=0.2]{Bilder/Peak1.png}}
\only<4>{\includegraphics[scale=0.2]{Bilder/Peak2.png}}
\only<5>{\includegraphics[scale=0.2]{Bilder/PeakPeak.png}}
}
\end{frame}

\begin{frame}{Frequenzen}
\begin{itemize}
\item<1-> Chirp Frequenz
	\begin{itemize}
	\item<2->[$\rightarrow$] Anzahl der Peaks pro Sekunde
	\end{itemize}
\item<3-> Peak Frequenz
	\begin{itemize}
	\item<4->[$\rightarrow$] Anzahl der PeakPeaks pro Sekunde
	\end{itemize}
\end{itemize}
\end{frame}

\subsection{Simulation}
\begin{frame}{Simulation}
\includegraphics[scale=0.5]{Bilder/Simulation.png}
\end{frame}
%-------------------------------------------------------------------------------------------------
\section{Physikalische Analyse}
\subsection{Chirp}
\subsubsection{Physikalische Beschreibung des Tons}

\begin{frame}{Periodendauer}
\begin{columns}
\begin{column}{.67\textwidth}
	\includegraphics[scale=0.3]{Bilder/Periodendauer.jpg}
\end{column}
\begin{column}{.33\textwidth}
	\begin{itemize}
	\item<2-> $\delta_n=\delta_1\cdot b^{n-1}$ mit $0\leq b<1$
	\item[ ] \ %just for moving the line up ^_^
	\end{itemize}
\end{column}
\end{columns}
\end{frame}

\begin{frame}{Amplitude}
\begin{columns}
\begin{column}{.67\textwidth}
	\includegraphics[scale=0.3]{Bilder/Amplitude.jpg}
\end{column}
\begin{column}{.33\textwidth}
	\begin{itemize}
	\item<2-> $y_n=a\cdot y_{n-1}$ mit $0\leq a<1$
	\item<3-> $\Leftrightarrow y_n=a^n\cdot y_0$
	\end{itemize}
\end{column}
\end{columns}
	
\end{frame}

\begin{frame}{Gleichsetzung}
\begin{itemize}
\item<1-> beides nach $n$ umformen:
	\begin{itemize}
	\item<2->[$\rightarrow$] Periode: $n=\frac{\log\left(1-\frac{t_n}{t_{ges}}\right)}{\log b}$
	\item<3->[$\rightarrow$] Amplitude: $n=\frac{\log\frac{y_n}{y_0}}{\log a}$
	\end{itemize}
\item<4-> Gleichsetzen und nach $y_n$ umformen:
	\begin{itemize}
	\item<5->[$\rightarrow$] $y_n=y_0\cdot\left(1-\frac{t_n}{t_{ges}}\right)^{\frac{\log a}{\log b}}$
	\end{itemize}
\end{itemize}
\end{frame}

\subsubsection{Verallgemeinerung}
\begin{frame}{Verallgemeinerung - warum}
\begin{itemize}
\item<1-> unterschiedliche rücktreibende Kräfte \only<5->{$\rightarrow$ Potenz des Weges}
	\begin{itemize}
	\item<2->[-] Gravitation \only<6->{$\rightarrow$ nahezu $s^0$}
	\item<3->[-] Magnetkraft \only<7->{$\rightarrow$ $s^{-2}$ (homogen)}
	\item<4->[-] Federkaft \only<8->{$\rightarrow$ $s^1$}
	\end{itemize}
\end{itemize}
\end{frame}

\begin{frame}{Verallgemeinerung - Ansatz}
\begin{itemize}
\item<1-> $F=c\cdot s^a$
\item<2->[$\rightarrow$] $E=\frac{c}{a+1}\cdot s^{a+1}$
\item<3-> Schwingung zwischen dieser und kinetischer Energie
	\begin{itemize}
	\item<4->[$\rightarrow$] $\frac{m}{2}v^2+\frac{c}{a+1}\cdot s^{a+1}=konst$
	\end{itemize}
\end{itemize}
\end{frame}

\begin{frame}{Verallgemeinerung - Ansatz}
\begin{itemize}
\item<1-> \textbf{Durch Ableiten:}
\item<2->[ ] $\frac{m\cdot 2v\cdot \dot{v}}{2}+c\cdot s^a\cdot\dot{s}=0$
\item<3->[$\Leftrightarrow$] $m\cdot \ddot{s}+c\cdot s^a=0$
\end{itemize}
\end{frame}

\begin{frame}{Verallgemeinerung - Nummerische Simulation}
\begin{itemize}
\item<1-> nicht-lineare Differentialgleichung
\item<2-> In "`Simulation"'Energieverlust durch Abnahme von $v$ implementiert
\item<3-> erste Nullstelle entspricht der Zeitspanne bis zum nächsten Peak
\end{itemize}
\end{frame}

%jeden graphen einzeln anzeigen
\begin{frame}{Verallgemeinerung - Nummerische Simulation}
\only<1>{\includegraphics[scale=0.2]{Bilder/Diff'Gleichung(0).jpg}}
\only<2>{\includegraphics[scale=0.2]{Bilder/Diff'Gleichung(01).jpg}}
\only<3>{\includegraphics[scale=0.2]{Bilder/Diff'Gleichung(012).jpg}}
\end{frame}

\subsection{Peak}
\subsubsection{Frequenz}
\begin{frame}{Frequenz}
\begin{itemize}
\item<1-> Peak Frequenz für jeden Peak gleich
\item<2-> zwei Ursprünge:
	\begin{itemize}
	\item<3-> Eigenfrequenz der Kugeln
	\item<4-> Frequenz zwischen den Kugeln
	\end{itemize}
\end{itemize}
\end{frame}

\begin{frame}{Eigenfrequenz}
\begin{itemize}
\item<1-> stehende Welle in den Kugeln
	\begin{itemize}
	\item<2->[$\rightarrow$] $f=\frac{c}{\lambda}=\frac{5170\sfrac{m}{s}}{8\cdot 0,017m}\approx 38kHz$
	\item<3->[$\rightarrow$] nicht hörbar
	\end{itemize}
\item<4-> andere Wellenlängen messbar aber nicht hörbar
\end{itemize}
\end{frame}

\begin{frame}{"`Auftreff Frequenz"'}
\begin{itemize}
\item<1-> Arbeit von K. Mehraby, H Khadem-hosseini Beheshti und M. Poursina\footnotemark
	\begin{itemize}
	\item<2->[$\rightarrow$] $f=\frac{76,1}{r}Hz=\frac{76,1}{0,017}Hz\approx 4476Hz$
	\end{itemize}
\end{itemize}
\footnotetext[1]{\hyperlink{http://www.researchgate.net/profile/Mehrdad_Poursina/publication/225820012_Impact_noise_radiated_by_collision_of_two_spheres_Comparison_between_numerical_simulations_experiments_and_analytical_results/links/00463527e9e985bfd2000000}{\tiny Impact noise radiated by collision of two spheres: Comparison between numerical\\
\vspace{-3pt}\hspace{18pt}simulations, experiments and analytical results}}
\end{frame}

\begin{frame}{Frequenzanalyse}
\begin{columns}
\begin{column}{.6\textwidth}
\includegraphics[scale=0.28]{Bilder/Frequenzen.jpg}
\end{column}
\begin{column}{.4\textwidth}
\begin{itemize}
\item<2-> hörbare Frequenz $4406Hz$ "`Auftreff Frequenz"'
\item<3-> nicht hörbare Frequenz $39kHz$ Eigenfrequenz
\item[ ] \ 
\end{itemize}
\end{column}
\end{columns}
\end{frame}

\begin{frame}
\center
{\HUGE Vielen Dank für Ihre Aufmerksamkeit}
\end{frame}
\end{document}