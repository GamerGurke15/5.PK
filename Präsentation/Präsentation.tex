\iffalse
TODO:

- Magneten kaufen: http://www.amazon.de/Singende-Magnetsteine-Rattle-Snake-2er/dp/B00511MMY0/ref=sr_1_2?ie=UTF8&qid=1426673724&sr=8-2&keywords=zwitscher+magnete

- Datum

- Chirp center

- Simulation alt tab

Ablauf:
1. Experiment + Erklärung + Simulation
2. Analyse erklären
	a. Chirp, Peak, PeakPeak
	b. Trennung von Frequenzen
		1. Chirp Frequenz (wann die Peaks kommen)
		2. Peak Frequenz (die Frequenz eines einzelnen Peaks)
3. Analyse
	a. Chirp Analyse
		1. Periodendauer ((Zeitabstände) (1/f)) --> e Funktion, Faktor wegen damped oszilation
		2. Amplitude über Zeit siehe oben, gleicher Faktor (Ungenauigkeiten)
		3. Gleichsetzung (Seite 10)
			a. Ansätze von den einzelnen Faktoren
			b. Ergebnisse
		4. Verallgemeinerung
			a. why
			b. Energie Ansatz
			c. Graph
	b. Peak Analyse
		1. Frequenz eines Peaks (siehe Nacharbeitung)
\fi


\documentclass[11pt]{beamer}
\usetheme{Warsaw}
\usepackage[utf8]{inputenc}
\usepackage[german]{babel}
\usepackage{amsmath}
\usepackage{amsfonts}
\usepackage{amssymb}
\usepackage{graphicx}
\usepackage{media9}
\usepackage{hyperref}
\usepackage{lastpage}
\usepackage{calc}
\author{Leonard Hackel und Niklas Schelten}
\title{5. PK - Ball Sound}
\subtitle{Wie entwickelt sich der Ton, der beim Zusammenstoß zweier Metallkugeln entsteht?}
\setbeamertemplate{navigation symbols}{}
\logo{\includegraphics[scale=0.028]{Bilder/Logo.png}} 
\institute{Herder Oberschule Berlin} 
\date{\today}
\subject{Physik} 

\newcommand{\multiplelines}[2][c]{\begin{tabular}[#1]{@{}l@{}}#2\end{tabular}}

%Frame numbers
\setbeamertemplate{footline}
{%
\begin{beamercolorbox}[wd=0.5\textwidth,ht=3ex,dp=1.5ex,leftskip=.5em,rightskip=.5em]{author in head/foot}%
\usebeamerfont{author in head/foot}%
\insertframenumber\ von \inserttotalframenumber\hfill\insertshortauthor%
\end{beamercolorbox}%
\vspace*{-4.5ex}\hspace*{0.5\textwidth}%
\begin{beamercolorbox}[wd=0.5\textwidth,ht=3ex,dp=1.5ex,left,leftskip=.5em]{title in head/foot}%
\usebeamerfont{title in head/foot}%
\insertshorttitle%
\end{beamercolorbox}%
}
%-------------------------------------------------------------------------------------------------
\begin{document}

\begin{frame}
\titlepage
\end{frame}

\begin{frame}
\tableofcontents
\end{frame}

%-------------------------------------------------------------------------------------------------
\section{Das Experiment}
\subsection{Vorführung}
\begin{frame}{Experiment}

\end{frame}

\subsection{Zusammensetzung des Tons}
\begin{frame}{\only<1-2>{Chirp}\only<3-4>{Peak}\only<5>{PeakPeak}}
\center{
\only<1>{\includemedia[addresource=Daten/Chirp.mp3, flashvars={source=Daten/Chirp.mp3&autoPlay=true}]
	{\includegraphics[scale=0.2]{Bilder/Chirp1.PNG}}{APlayer.swf}}
\only<2>{\includegraphics[scale=0.2]{Bilder/Chirp2.PNG}}
\only<3>{\includegraphics[scale=0.2]{Bilder/Peak1.PNG}}
\only<4>{\includegraphics[scale=0.2]{Bilder/Peak2.PNG}}
\only<5>{\includegraphics[scale=0.2]{Bilder/PeakPeak.PNG}}
}
\end{frame}

\begin{frame}{Frequenzen}
\begin{itemize}
\item<1-> Chirp Frequenz
	\begin{itemize}
	\item<2->[$\rightarrow$] Anzahl der Peaks pro Sekunde
	\end{itemize}
\item<3-> Peak Frequenz
	\begin{itemize}
	\item<4->[$\rightarrow$] Anzahl der PeakPeaks pro Sekunde
	\end{itemize}
\end{itemize}
\end{frame}

\subsection{Simulation}
\begin{frame}{Simulation}
\center
Alt Tab
\end{frame}
%-------------------------------------------------------------------------------------------------
\section{Physikalische Analyse}
\begin{frame}

\end{frame}
\subsection{Chirp}
\begin{frame}

\end{frame}
\subsection{Peak}
\begin{frame}

\end{frame}
\end{document}